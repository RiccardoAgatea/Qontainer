\subsection{Ripartizione ore} % (fold)
\label{sub:ripartizione_ore}
\begin{itemize}
	\item Analisi preliminare del problema: 1
	\item Progettazione modello: 2
	\begin{itemize}
		\item Progettazione template di classe \code{Container}: 0.5
		\item Progettazione gerarchia di classi: 1.5
	\end{itemize}
	\item Progettazione GUI: 1.5
	\item Apprendimento libreria Qt: 2.5
	\item Codifica modello: 22
	\begin{itemize}
		\item Codifica template di classe \code{Container}: 10.5
		\item Codifica gerarchia di classi: 11.5
	\end{itemize}
	\item Codifica GUI: 20
	\item Debugging: 2
	\item Testing: 1
\end{itemize}
\begin{nota}
	Le ore di apprendimento della libreria Qt si riferiscono al tempo speso per imparare ad utilizzare i sistemi e meccanismi propri della libreria (nello specifico la parent/child relationship, il meccanismo di slot e signal, il project file, e il resource system). Il tempo speso per prendere familiarità con le classi che la libreria fornisce per costruire la GUI è stato considerato come parte del tempo di codifica della stessa, in quanto difficilmente separabile da esso, ma nel complesso è non inferiore alle 5 ore.
\end{nota}
% subsection ripartizione_ore (end)
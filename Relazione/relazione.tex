\documentclass[10pt]{article}

\usepackage[italian]{babel}
\usepackage{fontspec}
\usepackage[dvipsnames]{xcolor}
\usepackage{tabularx}
\usepackage[margin=2cm]{geometry}
\usepackage{float}
\usepackage{graphicx}
\usepackage{minted}
\usepackage{amsthm}

\theoremstyle{remark}
\newtheorem*{nota}{Nota}

\newcommand{\code}[1]{\mintinline{cpp}{#1}}
\newcommand{\xmlcode}[1]{\mintinline{xml}{#1}}

\title{Progetto di Programmazione a Oggetti\\Qontainer}
\author{Agatea Riccardo, matricola 1170718}
\date{a.a. 2018/2019}

\begin{document}
\pagenumbering{arabic}
\maketitle
\section{Introduzione} % (fold)
\label{sec:introduzione}
L'applicazione consiste in un sistema di gestione delle ordinazioni di un ristorante; permette l'immissione di nuove ordinazioni, la loro modifica e rimozione, il loro salvataggio su file e caricamento da file, e la loro ricerca secondo diversi parametri. Inoltre, gestisce la separazione fra ordinazioni in attesa e completate.
% section introduzione (end)
\section{Descrizione aspetti progettuali} % (fold)
\label{sec:descrizione_aspetti_progettuali}
\subsection{Gerarchia di tipi} % (fold)
\label{sub:gerarchia_di_tipi}
La gerarchia modella le ordinazioni inviate alla cucina del ristorante. Radicata nella classe polimorfa astratta \code{Order}, si dirama in due direzioni, \code{Food} e \code{Drink}, associate rispettivamente a ordinazioni di piatti e bevande. Il polimorfismo è dato da alcuni metodi virtuali, descritti in seguito, che forniscono funzionalità di copia, move e confronto polimorfi, e permettono di ricavare informazioni esplicite sul tipo degli oggetti. La gerarchia è fortemente estensibile sia "in orizzontale", aggiungendo nuovi sottotipi alle classi base presenti, sia "in verticale", fornendo sottotipi alle classi più derivate della gerarchia.
\paragraph{Grafico della gerarchia} % (fold)
\label{par:grafico_della_gerarchia}

% paragraph grafico_della_gerarchia (end)
\paragraph{\code{Order}} % (fold)
\label{par:order}
La classe \code{Order} incapsula le caratteristiche comuni a tutte le ordinazioni, quali numero del tavolo e nome della pietanza ordinata. I campi dati, tutti privati, sono:
\begin{itemize}
	\item \code{unsigned int table}: rappresenta il numero del avolo da cui proviene l'ordinazione
	\item \code{std::string item}: rappresenta il nome della pietanza ordinata. Non sono effettuati controlli di coerenza con un eventuale listino o menu: si assume che tali controlli siano effettuati dall'utente della gerarchia.
	\item \code{static char separator}: alcuni metodi espongono all'utente stringhe formate dalla concatenazione di dettagli diversi dell'oggetto di invocazione; per mantenere la separazione delle varie caratteristiche viene utilizzato questo carattere.
	\item \code{static std::string empty}: gli stessi metodi possono indicare un valore "vuoto" per alcune caratteristiche attraverso questa stringa.
\end{itemize}
I metodi sono:
\begin{itemize}
	\item \code{protected}:
	\begin{itemize}
		\item \code{std::string getType() const}: metodo virtuale puro che per un'invocazione \code{p->getType()} ritorna il tipo di \code{*p} sottoforma di stringa. È pressochè equivalente a \code{typeid(*p).name()}, garantendo però che la stringa ritornata non sia implementation defined.
	\end{itemize}
	\item \code{public}:
	\begin{itemize}
		\item \code{Order(unsigned int, const std::string &)}: costruttore. Il primo parametro rappresenta il numero del tavolo, il secondo la pietanza ordinata.
		\item \code{Order(const Order &)}: costruttore di copia (standard).
		\item \code{Order(Order &&)}: costruttore di move (standard).
		\item \code{~Order()}: distruttore (standard). Ridefinito perchè sia virtuale.
		\item \code{Order& operator=(const Order &)}: operatore di assegnazione di copia (standard).
		\item \code{Order& operator=(Order &&)}: operatore di assegnazione di move (standard).
		\item \code{Order *clone() const}: metodo virtuale puro che implementa la "costruzione di copia polimorfa".
		\item \code{Order *move() = 0}: metodo virtuale puro analogo al precedente che implementa la "costruzione di move polimorfa".
		\item \code{std::string recap() const}: metodo che per un'invocazione \code{p->recap()} espone all'utente lo stato dell'oggetto di invocazione. La stringa ritornata riporta, nell'ordine, il tipo di \code{*p}, i dettagli (o eventualmente \code{empty}), il nome della pietanza ordinata, e il numero del tavolo, separati da \code{separator}. Non è virtuale, ma esegue chiamate a metodi virtuali.
		\item \code{std::string getDetails() const}: metodo virtuale puro che per un'invocazione \code{p->getDetails()} ritorna i dettagli dell'ordinazione \code{*p}, qualunque essi siano, in una stringa formattata in modo analogo a quella ritornata dal metodo recap().
		\item \code{void setDetails(const std::string &)}: metodo virtuale puro che per un'invocazione \code{p->setDetails()} modifica i dettagli dell'ordinazione \code{*p} in base al parametro passato al metodo.
		\item \code{bool operator==(const Order &) const}: operatore di uguaglianza, virtuale, \code{o1==o2} ritorna \code{true} se e solo se i due oggetti sono dello stesso tipo e i campi dati corrispondenti dei due oggetti sono uguali. Fornisce un'implementazione che esegue il confronto fra i tipi (dinamici) dei due oggetti confrontati, e fra i campi dati \code{table} e \code{item}.
		\item \code{bool operator!=(const Order &) const}: operatore di disuguaglianza, non virtuale ma chiama metodi virtuali, \code{o1!=o2} ritorna \code{true} se e solo se \code{!(o1==o2)} ritorna \code{true}.
	\end{itemize}
\end{itemize}
Sono inoltre fornite specializzazioni al tipo \code{Order} per i due template di funzione \code{template<typename T> T *clone(const T &)} e \code{template<typename T> T *clone(T &&)} definiti del namespace \code{PolyConstruct}. Per una reference \code{const Order &lref} e una reference a r-value \code{Order &&rref}, le chiamate \code{clone(lref)} e \code{clone(rref)} coincidono rispettivamente con \code{lref.clone()} e \code{rref.move()}.
% paragraph order (end)
\paragraph{\code{Food}} % (fold)
\label{par:food}
La classe \code{Food}, astratta, rappresenta le ordinazioni di cibo (in opposizione alle bevande). L'unico campo dati è:
\begin{itemize}
	\item \code{std::string without}: rappresenta eventuali rimozioni richieste dal cliente rispetto alla ricetta usuale
\end{itemize}
Gli unici metodi aggiuntivi sono il costruttore e i cinque metodi standard forniti dal compilatore. Sono forniti overriding per:
\begin{itemize}
	\item \code{std::string getDetails() const}: ritorna il contenuto di \code{without}.
	\item \code{void setDetails(const std::string &)}: assegna a without il valore del parametro.
	\item \code{bool operator==(const Order &) const}: dopo aver invocato esplicitamente \code{Order::operator==(const Order &)}, confronta i campi dati \code{without} dei due oggetti confrontati.
\end{itemize}
% paragraph food (end)
% subsection gerarchia_di_tipi (end)
\subsection{Template Container} % (fold)
\label{sub:template_container}
Le istanze del template di classe \code{Container<T>} sono contenitori che permettono di gestire collezioni di oggetti polimorfi (in caso non lo siano il template è comunque istanziabile, ma è necessario fornire specializzazioni ai template di funzione contenuti nel namespace \code{UniformInterface}, e il contenitore risultante è inutilmente appesantito). Siccome l'applicazione richiede che si possano eseguire inserimenti e rimozioni in posizione arbitraria nel container, si è scelto in fase di progettazione di utilizzare una lista doppiamente concatenata; si è inoltre scelto di fornire un'interfaccia pubblica il più simile possibile a quella del template \code{std::list<T>}.
\subsubsection{Classi annidate} % (fold)
\label{ssub:classi_annidate}
Essendo una lista concatenata, il template di classe si appoggia su un template di struttura annidata \code{Node}, che presenta diversi costruttori adatti ai diversi modi di passare gli oggetti da aggiungere al container come parametri, il distruttore, ed un overloading per l'operatore di uguaglianza, che esegue il confronto fra i campi info dei nodi considerati e dei successivi. Sono inoltre forniti due template di classe annidati \code{iterator<T>} e \code{const_iterator<T>}; in realtà, questi sono specializzazioni parziali del template di classe annidata \code{temp_iterator<T,constness>}, il cui parametro non-tipo \code{constness} è \code{true} per \code{const_iterator<T>} e \code{false} per \code{iterator<T>}. Questo, combinato con i template \code{reference<T,constness>} e \code{pointer<T,constness>} contenuti nel namespace \code{ReferenceTypes}, permette di evitare definire due classi separate, ma fortemente accoppiate, per gli iteratori e gli iteratori costanti, al costo di non permettere il cast da \code{iterator<T>} a \code{const_iterator<T>}. È fornito un costruttore privato ad un parametro, e per questo il template \code{Container} è dichiarato come template di classe friend associato, mentre il costruttore di default (che costruisce un iteratore non dereferenziabile) è pubblico; inoltre, sono presenti (e pubblici) i costruttori di copia e di move forniti dal compilatore, come anche i rispettivi operatori di assegnazione e il distruttore. Gli iteratori forniscono overloading per gli operatori di dereferenziazione, dereferenziazione e selezione, incremento e decremento prefisso e postfisso, e confronto. Il confronto è superficiale: l'operatore di uguaglianza ritorna \code{true} se e solo se i due iteratori puntano allo stesso nodo. In caso di comportamenti anomali, i metodi lanciano delle appropriate eccezioni.
% subsubsection classi_annidate (end)
\subsubsection{Metodi} % (fold)
\label{ssub:metodi}
\paragraph{Costruttori, Distuttore, Assegnazione} % (fold)
\label{par:costruttori_distuttore_assegnazione}
Il container fornisce 6 costruttori, di cui due sono il costruttore di copia e quello di move, i quali sono accompagnati dal distruttore e dagli operatori di assegnazione di copia e di move per la rule of five, che dallo standard C++11 ha sostituito la rule of three. I rimanenti costruttori permettono di costruire un container vuoto, un container di una data dimensione con nodi tutti uguali, un container a partire da un range, oppure un container a partire da una initializer list.
% paragraph costruttori_distuttore_assegnazione (end)
\paragraph{Iterazione} % (fold)
\label{par:iterazione}
Per l'iterazione sono forniti 6 metodi, completamente analoghi ai metodi \code{begin()}, \code{begin() const}, \code{cbegin() const}, \code{end()}, \code{end() const} e \code{cend() const} forniti dai container della STL. È inoltre fornito un metodo statico \code{toConstIter()} per convertire da \code{iterator<T>} a \code{const_iterator<T>}.
% paragraph iterazione (end)
\paragraph{Dimensione, Accesso, Inserimento, Rimozione} % (fold)
\label{par:dimensione_accesso_inserimento_rimozione}
Sono forniti un metodo \code{size()} che ritorna la dimensione della lista, e un metodo \code{empty()} che ritorna \code{true} se e solo se la lista è vuota, e metodi \code{front()} e \code{back()}, \code{const} e non, per l'accesso al primo e all'ultimo elemento. Sono forniti metodi \code{push_back()}, \code{push_front()}, \code{pop_back()}, e \code{pop_front()} per l'inserimento e la rimozione in testa e in coda alla lista (ciascuno dotato di diversi overloading),ed inoltre metodi \code{insert()} ed \code{erase()} per l'inserimento e la rimozione in posizione arbitraria, basati su iteratori. È fornito un metodo \code{clear()} per svuotare il container. Infine, sono forniti metodi \code{swap()} (in due versioni, per scambiare il contenuto di due container o di due nodi) e \code{give} (per spostare un nodo da un container ad un altro, o da un punto ad un altro nello stesso container);
% paragraph dimensione_accesso_inserimento_rimozione (end)
\paragraph{Ricerca, Confronto} % (fold)
\label{par:ricerca_confronto}
Sono forniti metodi \code{find()} e \code{find_if}, \code{const} e non, per la ricerca di elementi nel container (\code{find()} utilizza l'operatore di confronto fornito da \code{DeepPtr<T>}, mentre \code{find_if()} utilizza una funzione passata come parametro, la quale deve avere due parametri di tipo \code{const T &} e tipo di ritorno \code{bool}). Sono forniti inoltre operatori di confronto; l'operatore di uguaglianza ritorna \code{true} se e solo se i due container hanno gli stessi elementi nello stesso ordine.
% paragraph ricerca_confronto (end)
% subsubsection metodi (end)
% subsection template_container (end)
\subsection{Altre Classi} % (fold)
\label{sub:altre_classi}
Oltre alla gerarchia radicata in \code{Order} e al template \code{Container}, sono state definite altre classi.
\subsubsection{Template DeepPtr} % (fold)
\label{ssub:template_deepptr}
Il template di classe \code{DeepPtr<T>} è un template per puntatori smart. Gli oggetti delle classi istanziate da questo template utilizzano un campo dati di tipo \code{T *} per la gestione di oggetti anche polimorfi. La gestione profonda della memoria è garantita grazie a costruttore di copia, costruttore di move, operatore di assegnazione di copia, operatore di assegnazione di move, e distruttore. I rimanenti costruttori generano puntatori smart ad una copia dell'oggetto passato come parametro. Sono forniti operatori di dereferenziazione e di dereferenziazione e selezione, ciascuno in due versioni, \code{const} e non-\code{const}, ed operatori di confronto, che eseguono il confronto fra gli oggetti puntati. Sono inoltre forniti i metodi \code{swap()}, per scambiare i contenuti di due puntatori smart, e \code{takeResponsibility()}, per associare il puntatore smart ad un oggetto preesistente, invece di costruirne uno di copia. Per la copia e il confronto vengono utilizzati dei template di funzione contenuti nel namespace \code{UniformInterface}, che nella loro versione di default chiamano un metodo \code{clone()} e l'operatore di uguaglianza, ma che in caso di tipi privi di questi metodi permettono di essere adattati attraverso la specializzazione.
% subsubsection template_deepptr (end)
\subsubsection{Eccezioni} % (fold)
\label{ssub:eccezioni}
Sono state definite le classi \code{EmptyContainer}, \code{InvalidFile}, \code{InvalidIterator}, \code{NullPrtExcept}, e \code{UnavailableFile} per sollevare eccezioni. \code{InvalidFile} e \code{UnavailableFile} derivano da \code{std::invalid_argument}, mentre le restanti derivano da \code{std::logic_error}. Comunque tutte le classi sono parte della gerarchia radicata in \code{std::exception}.
% subsubsection eccezioni (end)
\subsubsection{Finestre di Dialogo} % (fold)
\label{ssub:finestre_di_dialogo}
Le classi \code{AddOrderDialog}, \code{EditOrderDialog}, e \code{SearchDialog} forniscono all'utente della GUI la possibilità di inserire informazioni, nello specifico per inserire nuovi ordini, modificare ordini esistenti, e selezionare ordini che rispettano specifiche caratteristiche. Sono tutte derivate da \code{QDialog}, in modo da rendere modali le finestre associate ai rispettivi oggetti. Tutte sfruttano \code{Order::info()} per rappresentare i dettagli specifici del tipo selezionato nel modo appropriato. In \code{AddOrderDialog} e \code{EditOrderDialog} le informazioni inserite dall'utente sono comunicate all'applicazione attraverso dei metodi essenzialmente analoghi a dei getter, mentre \code{SearchDialog} sfrutta due getter per comunicare se includere gli ordini in attesa e/o quelli completati, ed un metodo che ritorna un oggetto di tipo \code{std::function} che racchiude un predicato ad un parametro di tipo \code{const Order &} e ritorna \code{true} se e solo se il parametro rispetta le condizioni inserite dall'utente.
% subsubsection finestre_di_dialogo (end)
\subsubsection{OrderWidget} % (fold)
\label{ssub:orderwidget}
La classe \code{OrderWidget}, derivata da \code{QFrame}, permette di rappresentare gli ordini nella GUI. Contiene dei pulsanti che permettono all'utente di modificare l'ordine, completarlo o rimuoverlo, e sfrutta dei segnali per comunicare queste informazioni al resto dell'applicazione. Ogni oggetto di tipo \code{OrderWidget} contiene un indice (cioè un campo dati di tipo \code{Model::Index}, descritto in seguito) che punta al corrispettivo ordine. Come le finestre di dialogo, la classe sfrutta \code{Order::info()} per rappresentare i dettagli specifici dell'ordine nel modo appropriato. Ogni \code{OrderWidget} mostra un'icona appropriata al tipo dell'ordine associato, ed in caso di estensione della gerarchia è necessario aggiungere le icone necessarie attraverso il resource system di Qt. In particolare, l'immagine relativa al tipo \code{example} deve essere accessibile attraverso il prefisso \code{/type} e l'alias \code{example}: il path completo, indipendentemente dal nome del file, deve essere \code{:/type/example}.
% subsubsection orderwidget (end)
\subsubsection{Model, View, SearchView} % (fold)
\label{ssub:model_view_searchview}
Le classi \code{Model} e \code{View} costituiscono il fulcro dell'applicazione. \code{Model} racchiude la logica del programma, gestisce i due container di ordini in attesa e completati, permette di aggiungere ordini, rimuoverli, spostarli da un container all'altro, salvare su file e caricare da file, cercare all'interno dei due container ordini che rispettano certe caratteristiche, e ottenere l'elenco degli ordini incompleti. Espone il tipo \code{Model::Index}, che coincide con il tipo iteratore di \code{Container<Order>}, per permettere alla GUI di mantenere un collegamento diretto con gli ordini. \code{View}, classe derivata da \code{QMainWindow}, rappresenta la finestra principale dell'applicazione. È dotata di una toolbar per le operazioni eseguibili, ed utilizza una \code{QScrollArea} per visualizzare l'elenco degli ordini, ciascuno rappresentato da un \code{OrderWidget}. La classe \code{SearchView}, derivata da \code{QDialog} perchè sia modale, permette di visualizzare l'elenco degli ordini risultante da una ricerca. Inoltre, permette di rimuoverli o completarli tutti, chiedendo conferma all'utente attraverso un message box.
% subsubsection model_view_searchview (end)
% subsection altre_classi (end)
\subsection{Chiamate polimorfe} % (fold)
\label{sub:chiamate_polimorfe}
La gerarchia radicata in \code{Order} fornisce tre funzionalità diverse in modo polimorfico:
\begin{itemize}
	\item Clonazione
	\item RTTI su stringhe
	\item Gestione dei dettagli di un ordine
\end{itemize}
Queste funzionalità vengono sfruttate in parti diverse del progetto:
\begin{itemize}
	\item La clonazione viene utilizzata dai puntatori smart \code{DeepPtr}, che si appoggiano sulle funzioni racchiuse nel namespace \code{UniformInterface}. Le due funzioni clone invocano il metodo \code{clone()} nella loro implementazione di default.
	\item Il meccanismo di RTTI su stringhe viene utilizzato dalla classe \code{SearchDialog} per confrontare un ordine con i parametri della ricerca selezionati dall'utente, in particolare per verificare che l'ordine sia di un tipo selezionato, e che sia di un tipo che è derivato da almeno una delle classi astratte selezionate. Inoltre, la possibilità di ricavare il tipo di un oggetto in forma di stringa è utilizzata dalla classe \code{Model} per il salvataggio dei dati su file e dalla classe \code{OrderWidget} per mostrare il tipo delle ordinazioni nella GUI.
	\item La gestione dei dettagli viene utilizzata:
	\begin{itemize}
		\item "In lettura" (invocando \code{getDetails()}) da \code{Model} per il salvataggio dei dati, da \code{OrderWidget} per mostrare le informazioni nella GUI, e da \code{SearchDialog} per verificare che ciascun dettaglio di un ordine contenga i parametri di ricerca richiesti dall'utente.
		\item "In scrittura" (invocando \code{setDetails()}) da \code{OrderWidget} per sovrascrivere i dettagli in seguito ad una richiesta di modifica da parte dell'utente.
	\end{itemize}
\end{itemize}
% subsection chiamate_polimorfe (end)
\subsection{Formato dei file di salvataggio e caricamento} % (fold)
\label{sub:formato_dei_file_di_salvataggio_e_caricamento}

% subsection formato_dei_file_di_salvataggio_e_caricamento (end)
% section descrizione_aspetti_progettuali (end)
\section{Note tecniche} % (fold)
\label{sec:note_tecniche}
\subsection{Istruzioni di compilazione} % (fold)
\label{sub:istruzioni_di_compilazione}
Per la compilazione è fornito il file \code{progetto.pro}.
% subsection istruzioni_di_compilazione (end)
\subsection{Ambiente di sviluppo} % (fold)
\label{sub:ambiente_di_sviluppo}
\begin{itemize}
	\item Sistema operativo di sviluppo: Windows 10 Home 64-bit
	\item Compilatore: g++ (i686-posix-dwarf-rev0, Built by MinGW-W64 project) 5.3.0
	\item Qt framework: Qt 5.12.0
	\item IDE di sviluppo: Qt Creator 4.8.1
\end{itemize}
% subsection ambiente_di_sviluppo (end)
\subsection{Ripartizione ore} % (fold)
\label{sub:ripartizione_ore}
\begin{itemize}
	\item Analisi preliminare del problema: 1
	\item Progettazione modello: 1
	\begin{itemize}
		\item Progettazione template di classe \code{Container}: 0.5
		\item Progettazione gerarchia di classi: 1.5
	\end{itemize}
	\item Progettazione GUI: 1 (per ora)
	\item Apprendimento libreria Qt: 15 (da specificare, esterne alle ore del progetto, scrivi che hai smanettato per un altro progetto)
	\item Codifica modello: 17
	\begin{itemize}
		\item Codifica template di classe \code{Container}: 8
		\item Codifica gerarchia di classi: 9
	\end{itemize}
	\item Codifica GUI: (4) per ora
	\item Debugging
	\item Testing
\end{itemize}
% subsection ripartizione_ore (end)
% section note_tecniche (end)
\end{document}
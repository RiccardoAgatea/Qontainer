\documentclass{article}

\usepackage[italian]{babel}
\usepackage{fontspec}
\usepackage[dvipsnames]{xcolor}
\usepackage{tabularx}
\usepackage[margin=2cm]{geometry}
\usepackage{float}
\usepackage{graphicx}
\usepackage{minted}

\newcommand{\code}[1]{\mintinline{cpp}{#1}}

\title{Progetto di Programmazione a Oggetti}
\author{Agatea Riccardo}
\date{a.a. 2018/2019}

\begin{document}
\pagenumbering{gobble}
\maketitle
\newpage
\pagenumbering{arabic}
\section{Introduzione} % (fold)
\label{sec:introduzione}
L'applicazione consiste in un sistema di gestione delle ordinazioni di un ristorante. L'applicazione permette l'immissione di nuove ordinazioni, la loro modifica e rimozione, il loro salvataggio su file e caricamento da file in formato XML, e la loro ricerca secondo diversi parametri. Inoltre, gestisce la separazione fra ordinazioni in attesa e completate.
% section introduzione (end)
\section{Descrizione aspetti progettuali} % (fold)
\label{sec:descrizione_aspetti_progettuali}
\subsection{Gerarchia di tipi} % (fold)
\label{sub:gerarchia_di_tipi}
La gerarchia modella le ordinazioni inviate alla cucina del ristorante. Radicata nella classe polimorfa astratta \code{Order}, si dirama in due direzioni, \code{Food} e \code{Drink}, associate rispettivamente a ordinazioni di piatti e bevande. Il polimorfismo è dato da alcuni metodi virtuali, descritti in seguito, che forniscono funzionalità di copia, move e confronto polimorfi, e permettono di ricavare informazioni esplicite sul tipo degli oggetti. La gerarchia è fortemente estensibile sia "in orizzontale", aggiungendo nuovi sottotipi alle classi base presenti, sia "in verticale", fornendo sottotipi alle classi più derivate della gerarchia.
\paragraph{Grafico della gerarchia} % (fold)
\label{par:grafico_della_gerarchia}

% paragraph grafico_della_gerarchia (end)
\paragraph{\code{Order}} % (fold)
\label{par:order}
La classe \code{Order} incapsula le caratteristiche comuni a tutte le ordinazioni, quali numero del tavolo e nome della pietanza ordinata. I campi dati, tutti privati, sono:
\begin{itemize}
	\item \code{unsigned int table}: rappresenta il numero del avolo da cui proviene l'ordinazione
	\item \code{std::string item}: rappresenta il nome della pietanza ordinata. Non sono effettuati controlli di coerenza con un eventuale listino o menu: si assume che tali controlli siano effettuati dall'utente della gerarchia.
	\item \code{static char separator}: alcuni metodi espongono all'utente stringhe formate dalla concatenazione di dettagli diversi dell'oggetto di invocazione; per mantenere la separazione delle varie caratteristiche viene utilizzato questo carattere.
	\item \code{static std::string empty}: gli stessi metodi possono indicare un valore "vuoto" per alcune caratteristiche attraverso questa stringa.
\end{itemize}
I metodi sono:
\begin{itemize}
	\item \code{protected}:
	\begin{itemize}
		\item \code{std::string getType() const}: metodo virtuale puro che per un'invocazione \code{p->getType()} ritorna il tipo di \code{*p} sottoforma di stringa. È pressochè equivalente a \code{typeid(*p).name()}, garantendo però che la stringa ritornata non sia implementation defined.
	\end{itemize}
	\item \code{public}:
	\begin{itemize}
		\item \code{Order(unsigned int, const std::string &)}: costruttore. Il primo parametro rappresenta il numero del tavolo, il secondo la pietanza ordinata.
		\item \code{Order(const Order &)}: costruttore di copia (standard).
		\item \code{Order(Order &&)}: costruttore di move (standard).
		\item \code{~Order()}: distruttore (standard). Ridefinito perchè sia virtuale.
		\item \code{Order& operator=(const Order &)}: operatore di assegnazione di copia (standard).
		\item \code{Order& operator=(Order &&)}: operatore di assegnazione di move (standard).
		\item \code{Order *clone() const}: metodo virtuale puro che implementa la "costruzione di copia polimorfa".
		\item \code{Order *move() = 0}: metodo virtuale puro analogo al precedente che implementa la "costruzione di move polimorfa".
		\item \code{std::string recap() const}: metodo che per un'invocazione \code{p->recap()} espone all'utente lo stato dell'oggetto di invocazione. La stringa ritornata riporta, nell'ordine, il tipo di \code{*p}, i dettagli (o eventualmente \code{empty}), il nome della pietanza ordinata, e il numero del tavolo, separati da \code{separator}. Non è virtuale, ma esegue chiamate a metodi virtuali.
		\item \code{std::string getDetails() const}: metodo virtuale puro che per un'invocazione \code{p->getDetails()} ritorna i dettagli dell'ordinazione \code{*p}, qualunque essi siano, in una stringa formattata in modo analogo a quella ritornata dal metodo recap().
		\item \code{void setDetails(const std::string &)}: metodo virtuale puro che per un'invocazione \code{p->setDetails()} modifica i dettagli dell'ordinazione \code{*p} in base al parametro passato al metodo.
		\item \code{bool operator==(const Order &) const}: operatore di uguaglianza, virtuale, \code{o1==o2} ritorna \code{true} se e solo se i due oggetti sono dello stesso tipo e i campi dati corrispondenti dei due oggetti sono uguali. Fornisce un'implementazione che esegue il confronto fra i tipi (dinamici) dei due oggetti confrontati, e fra i campi dati \code{table} e \code{item}.
		\item \code{bool operator!=(const Order &) const}: operatore di disuguaglianza, non virtuale ma chiama metodi virtuali, \code{o1!=o2} ritorna \code{true} se e solo se \code{!(o1==o2)} ritorna \code{true}.
	\end{itemize}
\end{itemize}
Sono inoltre fornite specializzazioni al tipo \code{Order} per i due template di funzione \code{template<typename T> T *clone(const T &)} e \code{template<typename T> T *clone(T &&)} definiti del namespace \code{PolyConstruct}. Per una reference \code{const Order &lref} e una reference a r-value \code{Order &&rref}, le chiamate \code{clone(lref)} e \code{clone(rref)} coincidono rispettivamente con \code{lref.clone()} e \code{rref.move()}.
% paragraph order (end)
\paragraph{\code{Food}} % (fold)
\label{par:food}
La classe \code{Food}, astratta, rappresenta le ordinazioni di cibo (in opposizione alle bevande). L'unico campo dati è:
\begin{itemize}
	\item \code{std::string without}: rappresenta eventuali rimozioni richieste dal cliente rispetto alla ricetta usuale
\end{itemize}
Gli unici metodi aggiuntivi sono il costruttore e i cinque metodi standard forniti dal compilatore. Sono forniti overriding per:
\begin{itemize}
	\item \code{std::string getDetails() const}: ritorna il contenuto di \code{without}.
	\item \code{void setDetails(const std::string &)}: assegna a without il valore del parametro.
	\item \code{bool operator==(const Order &) const}: dopo aver invocato esplicitamente \code{Order::operator==(const Order &)}, confronta i campi dati \code{without} dei due oggetti confrontati.
\end{itemize}
% paragraph food (end)
% subsection gerarchia_di_tipi (end)
\subsection{Chiamate polimorfe} % (fold)
\label{sub:chiamate_polimorfe}
La gerarchia radicata in \code{Order} fornisce tre funzionalità diverse in modo polimorfico:
\begin{itemize}
	\item Clonazione
	\item RTTI su stringhe
	\item Gestione dei dettagli di un ordine
\end{itemize}
Queste funzionalità vengono sfruttate in parti diverse del progetto:
\begin{itemize}
	\item La clonazione viene utilizzata dai puntatori smart \code{DeepPtr}, che si appoggiano sulle funzioni racchiuse nel namespace \code{UniformInterface}. Le due funzioni clone invocano il metodo \code{clone()} nella loro implementazione di default.
	\item Il meccanismo di RTTI su stringhe viene utilizzato dalla classe \code{SearchDialog} per confrontare un ordine con i parametri della ricerca selezionati dall'utente, in particolare per verificare che l'ordine sia di un tipo selezionato, e che sia di un tipo che è derivato da almeno una delle classi astratte selezionate. Inoltre, la possibilità di ricavare il tipo di un oggetto in forma di stringa è utilizzata dalla classe \code{Model} per il salvataggio dei dati su file e dalla classe \code{OrderWidget} per mostrare il tipo delle ordinazioni nella GUI.
	\item La gestione dei dettagli viene utilizzata:
	\begin{itemize}
		\item "In lettura" (invocando \code{getDetails()}) da \code{Model} per il salvataggio dei dati, da \code{OrderWidget} per mostrare le informazioni nella GUI, e da \code{SearchDialog} per verificare che ciascun dettaglio di un ordine contenga i parametri di ricerca richiesti dall'utente.
		\item "In scrittura" (invocando \code{setDetails()}) da \code{OrderWidget} per sovrascrivere i dettagli in seguito ad una richiesta di modifica da parte dell'utente.
	\end{itemize}
\end{itemize}
% subsection chiamate_polimorfe (end)
\subsection{Formato dei file di salvataggio e caricamento} % (fold)
\label{sub:formato_dei_file_di_salvataggio_e_caricamento}

% subsection formato_dei_file_di_salvataggio_e_caricamento (end)
% section descrizione_aspetti_progettuali (end)
\section{Note tecniche} % (fold)
\label{sec:note_tecniche}
\subsection{Istruzioni di compilazione} % (fold)
\label{sub:istruzioni_di_compilazione}
Per la compilazione è fornito il file \code{progetto.pro}.
% subsection istruzioni_di_compilazione (end)
\subsection{Ambiente di sviluppo} % (fold)
\label{sub:ambiente_di_sviluppo}
\begin{itemize}
	\item Sistema operativo di sviluppo: Windows 10 Home 64-bit
	\item Compilatore: g++ (i686-posix-dwarf-rev0, Built by MinGW-W64 project) 5.3.0
	\item Qt framework: Qt 5.12.0
	\item IDE di sviluppo: Qt Creator 4.8.1
\end{itemize}
% subsection ambiente_di_sviluppo (end)
\subsection{Ripartizione ore} % (fold)
\label{sub:ripartizione_ore}
\begin{itemize}
	\item Analisi preliminare del problema: 1
	\item Progettazione modello: 1
	\begin{itemize}
		\item Progettazione template di classe \code{Container}: 0.5
		\item Progettazione gerarchia di classi: 1.5
	\end{itemize}
	\item Progettazione GUI: 1 (per ora)
	\item Apprendimento libreria Qt: 15 (da specificare, esterne alle ore del progetto, scrivi che hai smanettato per un altro progetto)
	\item Codifica modello: 17
	\begin{itemize}
		\item Codifica template di classe \code{Container}: 8
		\item Codifica gerarchia di classi: 9
	\end{itemize}
	\item Codifica GUI: (4) per ora
	\item Debugging
	\item Testing
\end{itemize}
% subsection ripartizione_ore (end)
% section note_tecniche (end)
\end{document}